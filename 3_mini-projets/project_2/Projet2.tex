% Options for packages loaded elsewhere
\PassOptionsToPackage{unicode}{hyperref}
\PassOptionsToPackage{hyphens}{url}
\documentclass[
]{article}
\usepackage{xcolor}
\usepackage[margin=1in]{geometry}
\usepackage{amsmath,amssymb}
\setcounter{secnumdepth}{-\maxdimen} % remove section numbering
\usepackage{iftex}
\ifPDFTeX
  \usepackage[T1]{fontenc}
  \usepackage[utf8]{inputenc}
  \usepackage{textcomp} % provide euro and other symbols
\else % if luatex or xetex
  \usepackage{unicode-math} % this also loads fontspec
  \defaultfontfeatures{Scale=MatchLowercase}
  \defaultfontfeatures[\rmfamily]{Ligatures=TeX,Scale=1}
\fi
\usepackage{lmodern}
\ifPDFTeX\else
  % xetex/luatex font selection
\fi
% Use upquote if available, for straight quotes in verbatim environments
\IfFileExists{upquote.sty}{\usepackage{upquote}}{}
\IfFileExists{microtype.sty}{% use microtype if available
  \usepackage[]{microtype}
  \UseMicrotypeSet[protrusion]{basicmath} % disable protrusion for tt fonts
}{}
\makeatletter
\@ifundefined{KOMAClassName}{% if non-KOMA class
  \IfFileExists{parskip.sty}{%
    \usepackage{parskip}
  }{% else
    \setlength{\parindent}{0pt}
    \setlength{\parskip}{6pt plus 2pt minus 1pt}}
}{% if KOMA class
  \KOMAoptions{parskip=half}}
\makeatother
\usepackage{color}
\usepackage{fancyvrb}
\newcommand{\VerbBar}{|}
\newcommand{\VERB}{\Verb[commandchars=\\\{\}]}
\DefineVerbatimEnvironment{Highlighting}{Verbatim}{commandchars=\\\{\}}
% Add ',fontsize=\small' for more characters per line
\usepackage{framed}
\definecolor{shadecolor}{RGB}{248,248,248}
\newenvironment{Shaded}{\begin{snugshade}}{\end{snugshade}}
\newcommand{\AlertTok}[1]{\textcolor[rgb]{0.94,0.16,0.16}{#1}}
\newcommand{\AnnotationTok}[1]{\textcolor[rgb]{0.56,0.35,0.01}{\textbf{\textit{#1}}}}
\newcommand{\AttributeTok}[1]{\textcolor[rgb]{0.13,0.29,0.53}{#1}}
\newcommand{\BaseNTok}[1]{\textcolor[rgb]{0.00,0.00,0.81}{#1}}
\newcommand{\BuiltInTok}[1]{#1}
\newcommand{\CharTok}[1]{\textcolor[rgb]{0.31,0.60,0.02}{#1}}
\newcommand{\CommentTok}[1]{\textcolor[rgb]{0.56,0.35,0.01}{\textit{#1}}}
\newcommand{\CommentVarTok}[1]{\textcolor[rgb]{0.56,0.35,0.01}{\textbf{\textit{#1}}}}
\newcommand{\ConstantTok}[1]{\textcolor[rgb]{0.56,0.35,0.01}{#1}}
\newcommand{\ControlFlowTok}[1]{\textcolor[rgb]{0.13,0.29,0.53}{\textbf{#1}}}
\newcommand{\DataTypeTok}[1]{\textcolor[rgb]{0.13,0.29,0.53}{#1}}
\newcommand{\DecValTok}[1]{\textcolor[rgb]{0.00,0.00,0.81}{#1}}
\newcommand{\DocumentationTok}[1]{\textcolor[rgb]{0.56,0.35,0.01}{\textbf{\textit{#1}}}}
\newcommand{\ErrorTok}[1]{\textcolor[rgb]{0.64,0.00,0.00}{\textbf{#1}}}
\newcommand{\ExtensionTok}[1]{#1}
\newcommand{\FloatTok}[1]{\textcolor[rgb]{0.00,0.00,0.81}{#1}}
\newcommand{\FunctionTok}[1]{\textcolor[rgb]{0.13,0.29,0.53}{\textbf{#1}}}
\newcommand{\ImportTok}[1]{#1}
\newcommand{\InformationTok}[1]{\textcolor[rgb]{0.56,0.35,0.01}{\textbf{\textit{#1}}}}
\newcommand{\KeywordTok}[1]{\textcolor[rgb]{0.13,0.29,0.53}{\textbf{#1}}}
\newcommand{\NormalTok}[1]{#1}
\newcommand{\OperatorTok}[1]{\textcolor[rgb]{0.81,0.36,0.00}{\textbf{#1}}}
\newcommand{\OtherTok}[1]{\textcolor[rgb]{0.56,0.35,0.01}{#1}}
\newcommand{\PreprocessorTok}[1]{\textcolor[rgb]{0.56,0.35,0.01}{\textit{#1}}}
\newcommand{\RegionMarkerTok}[1]{#1}
\newcommand{\SpecialCharTok}[1]{\textcolor[rgb]{0.81,0.36,0.00}{\textbf{#1}}}
\newcommand{\SpecialStringTok}[1]{\textcolor[rgb]{0.31,0.60,0.02}{#1}}
\newcommand{\StringTok}[1]{\textcolor[rgb]{0.31,0.60,0.02}{#1}}
\newcommand{\VariableTok}[1]{\textcolor[rgb]{0.00,0.00,0.00}{#1}}
\newcommand{\VerbatimStringTok}[1]{\textcolor[rgb]{0.31,0.60,0.02}{#1}}
\newcommand{\WarningTok}[1]{\textcolor[rgb]{0.56,0.35,0.01}{\textbf{\textit{#1}}}}
\usepackage{graphicx}
\makeatletter
\newsavebox\pandoc@box
\newcommand*\pandocbounded[1]{% scales image to fit in text height/width
  \sbox\pandoc@box{#1}%
  \Gscale@div\@tempa{\textheight}{\dimexpr\ht\pandoc@box+\dp\pandoc@box\relax}%
  \Gscale@div\@tempb{\linewidth}{\wd\pandoc@box}%
  \ifdim\@tempb\p@<\@tempa\p@\let\@tempa\@tempb\fi% select the smaller of both
  \ifdim\@tempa\p@<\p@\scalebox{\@tempa}{\usebox\pandoc@box}%
  \else\usebox{\pandoc@box}%
  \fi%
}
% Set default figure placement to htbp
\def\fps@figure{htbp}
\makeatother
\setlength{\emergencystretch}{3em} % prevent overfull lines
\providecommand{\tightlist}{%
  \setlength{\itemsep}{0pt}\setlength{\parskip}{0pt}}
\usepackage{bookmark}
\IfFileExists{xurl.sty}{\usepackage{xurl}}{} % add URL line breaks if available
\urlstyle{same}
\hypersetup{
  pdftitle={Analyse de Sentiment - Twitter Data},
  pdfauthor={Tonlop Lucas},
  hidelinks,
  pdfcreator={LaTeX via pandoc}}

\title{Analyse de Sentiment - Twitter Data}
\author{Tonlop Lucas}
\date{2026-02-19}

\begin{document}
\maketitle

\subsection{1. Chargement des données}\label{chargement-des-donnuxe9es}

On charge les datasets d'entraînement et de validation fournis, en
spécifiant les noms des colonnes pour une meilleure lisibilité. On
définira les colonnes comme suit : ID, Entité, Sentiment, Contenu du
tweet.

\begin{Shaded}
\begin{Highlighting}[]
\NormalTok{train\_data }\OtherTok{\textless{}{-}} \FunctionTok{read\_csv}\NormalTok{(}\StringTok{"twitter\_training.csv"}\NormalTok{, }\AttributeTok{col\_names =} \FunctionTok{c}\NormalTok{(}\StringTok{"ID"}\NormalTok{, }\StringTok{"Entity"}\NormalTok{, }\StringTok{"Sentiment"}\NormalTok{, }\StringTok{"Tweet\_Content"}\NormalTok{))}
\end{Highlighting}
\end{Shaded}

\begin{verbatim}
## Rows: 74682 Columns: 4
## -- Column specification --------------------------------------------------------
## Delimiter: ","
## chr (3): Entity, Sentiment, Tweet_Content
## dbl (1): ID
## 
## i Use `spec()` to retrieve the full column specification for this data.
## i Specify the column types or set `show_col_types = FALSE` to quiet this message.
\end{verbatim}

\begin{Shaded}
\begin{Highlighting}[]
\NormalTok{val\_data }\OtherTok{\textless{}{-}} \FunctionTok{read\_csv}\NormalTok{(}\StringTok{"twitter\_validation.csv"}\NormalTok{, }\AttributeTok{col\_names =} \FunctionTok{c}\NormalTok{(}\StringTok{"ID"}\NormalTok{, }\StringTok{"Entity"}\NormalTok{, }\StringTok{"Sentiment"}\NormalTok{, }\StringTok{"Tweet\_Content"}\NormalTok{))}
\end{Highlighting}
\end{Shaded}

\begin{verbatim}
## Rows: 1000 Columns: 4
## -- Column specification --------------------------------------------------------
## Delimiter: ","
## chr (3): Entity, Sentiment, Tweet_Content
## dbl (1): ID
## 
## i Use `spec()` to retrieve the full column specification for this data.
## i Specify the column types or set `show_col_types = FALSE` to quiet this message.
\end{verbatim}

\begin{Shaded}
\begin{Highlighting}[]
\FunctionTok{head}\NormalTok{(train\_data)}
\end{Highlighting}
\end{Shaded}

\begin{verbatim}
## # A tibble: 6 x 4
##      ID Entity      Sentiment Tweet_Content                                     
##   <dbl> <chr>       <chr>     <chr>                                             
## 1  2401 Borderlands Positive  im getting on borderlands and i will murder you a~
## 2  2401 Borderlands Positive  I am coming to the borders and I will kill you al~
## 3  2401 Borderlands Positive  im getting on borderlands and i will kill you all,
## 4  2401 Borderlands Positive  im coming on borderlands and i will murder you al~
## 5  2401 Borderlands Positive  im getting on borderlands 2 and i will murder you~
## 6  2401 Borderlands Positive  im getting into borderlands and i can murder you ~
\end{verbatim}

\begin{Shaded}
\begin{Highlighting}[]
\CommentTok{\#on verifie la taille de nos datasets et si on a du texte manquant}
\CommentTok{\# Nombre de lignes}
\FunctionTok{nrow}\NormalTok{(train\_data)}
\end{Highlighting}
\end{Shaded}

\begin{verbatim}
## [1] 74682
\end{verbatim}

\begin{Shaded}
\begin{Highlighting}[]
\CommentTok{\# Vérification des valeurs manquantes}
\FunctionTok{sum}\NormalTok{(}\FunctionTok{is.na}\NormalTok{(train\_data}\SpecialCharTok{$}\NormalTok{Tweet\_Content))}
\end{Highlighting}
\end{Shaded}

\begin{verbatim}
## [1] 858
\end{verbatim}

\begin{Shaded}
\begin{Highlighting}[]
\CommentTok{\# Suppression des lignes sans contenu textuel pour l\textquotesingle{}analyse}
\NormalTok{train\_data }\OtherTok{\textless{}{-}}\NormalTok{ train\_data }\SpecialCharTok{\%\textgreater{}\%} \FunctionTok{filter}\NormalTok{(}\SpecialCharTok{!}\FunctionTok{is.na}\NormalTok{(Tweet\_Content))}

\CommentTok{\#on regarde comment sont réparties les classes dans notre dataset d\textquotesingle{}entrainement}
\FunctionTok{ggplot}\NormalTok{(train\_data, }\FunctionTok{aes}\NormalTok{(}\AttributeTok{x =}\NormalTok{ Sentiment, }\AttributeTok{fill =}\NormalTok{ Sentiment)) }\SpecialCharTok{+}
  \FunctionTok{geom\_bar}\NormalTok{() }\SpecialCharTok{+}
  \FunctionTok{theme\_minimal}\NormalTok{() }\SpecialCharTok{+}
  \FunctionTok{labs}\NormalTok{(}\AttributeTok{title =} \StringTok{"Répartition globale des Sentiments"}\NormalTok{,}
       \AttributeTok{x =} \StringTok{"Catégorie de Sentiment"}\NormalTok{,}
       \AttributeTok{y =} \StringTok{"Nombre de Tweets"}\NormalTok{) }\SpecialCharTok{+}
  \FunctionTok{scale\_fill\_brewer}\NormalTok{(}\AttributeTok{palette =} \StringTok{"Set2"}\NormalTok{)}
\end{Highlighting}
\end{Shaded}

\pandocbounded{\includegraphics[keepaspectratio]{Projet2_files/figure-latex/unnamed-chunk-1-1.pdf}}

\subsection{1.2 Nettoyage des données}\label{nettoyage-des-donnuxe9es}

On procède au nettoyage des tweets en supprimant les URLs, la
ponctuation, les chiffres, les caractères spéciaux, et en convertissant
le texte en minuscules.

\begin{Shaded}
\begin{Highlighting}[]
\FunctionTok{library}\NormalTok{(stringr)}
\FunctionTok{library}\NormalTok{(tidytext)}

\NormalTok{clean\_tweets }\OtherTok{\textless{}{-}} \ControlFlowTok{function}\NormalTok{(text) \{}
\NormalTok{  text }\OtherTok{\textless{}{-}} \FunctionTok{str\_to\_lower}\NormalTok{(text) }\CommentTok{\# Tout en minuscules}
\NormalTok{  text }\OtherTok{\textless{}{-}} \FunctionTok{str\_replace\_all}\NormalTok{(text, }\StringTok{"http}\SpecialCharTok{\textbackslash{}\textbackslash{}}\StringTok{S+|www}\SpecialCharTok{\textbackslash{}\textbackslash{}}\StringTok{S+|\&amp;"}\NormalTok{, }\StringTok{""}\NormalTok{) }\CommentTok{\# Supprimer URLs}
\NormalTok{  text }\OtherTok{\textless{}{-}} \FunctionTok{str\_replace\_all}\NormalTok{(text, }\StringTok{"[[:punct:]]"}\NormalTok{, }\StringTok{" "}\NormalTok{) }\CommentTok{\# Supprimer ponctuation}
\NormalTok{  text }\OtherTok{\textless{}{-}} \FunctionTok{str\_replace\_all}\NormalTok{(text, }\StringTok{"[[:digit:]]"}\NormalTok{, }\StringTok{""}\NormalTok{) }\CommentTok{\# Supprimer chiffres}
\NormalTok{  text }\OtherTok{\textless{}{-}} \FunctionTok{str\_replace\_all}\NormalTok{(text, }\StringTok{"[\^{}}\SpecialCharTok{\textbackslash{}\textbackslash{}}\StringTok{x01{-}}\SpecialCharTok{\textbackslash{}\textbackslash{}}\StringTok{x7F]"}\NormalTok{, }\StringTok{""}\NormalTok{) }\CommentTok{\# Supprimer caractères spéciaux non{-}ASCII}
\NormalTok{  text }\OtherTok{\textless{}{-}} \FunctionTok{str\_squish}\NormalTok{(text) }\CommentTok{\# Supprimer les espaces doubles ou inutiles}
  \FunctionTok{return}\NormalTok{(text)}
\NormalTok{\}}
\CommentTok{\# Application du nettoyage de base}
\NormalTok{train\_data }\OtherTok{\textless{}{-}}\NormalTok{ train\_data }\SpecialCharTok{\%\textgreater{}\%}
  \FunctionTok{mutate}\NormalTok{(}\AttributeTok{Tweet\_Clean =} \FunctionTok{clean\_tweets}\NormalTok{(Tweet\_Content))}

\CommentTok{\# Aperçu du résultat avant/après}
\FunctionTok{head}\NormalTok{(train\_data[}\FunctionTok{c}\NormalTok{(}\StringTok{"Tweet\_Content"}\NormalTok{, }\StringTok{"Tweet\_Clean"}\NormalTok{)], }\DecValTok{5}\NormalTok{)}
\end{Highlighting}
\end{Shaded}

\begin{verbatim}
## # A tibble: 5 x 2
##   Tweet_Content                                             Tweet_Clean         
##   <chr>                                                     <chr>               
## 1 im getting on borderlands and i will murder you all ,     im getting on borde~
## 2 I am coming to the borders and I will kill you all,       i am coming to the ~
## 3 im getting on borderlands and i will kill you all,        im getting on borde~
## 4 im coming on borderlands and i will murder you all,       im coming on border~
## 5 im getting on borderlands 2 and i will murder you me all, im getting on borde~
\end{verbatim}

\begin{Shaded}
\begin{Highlighting}[]
\CommentTok{\# Chargement de la liste des mots vides anglais}
\FunctionTok{data}\NormalTok{(}\StringTok{"stop\_words"}\NormalTok{)}

\CommentTok{\# Tokenisation (séparation mot par mot) et filtrage}
\NormalTok{train\_tokens }\OtherTok{\textless{}{-}}\NormalTok{ train\_data }\SpecialCharTok{\%\textgreater{}\%}
  \FunctionTok{unnest\_tokens}\NormalTok{(word, Tweet\_Clean) }\SpecialCharTok{\%\textgreater{}\%}
  \FunctionTok{anti\_join}\NormalTok{(stop\_words, }\AttributeTok{by =} \StringTok{"word"}\NormalTok{)}

\CommentTok{\# Visualisation des mots les plus fréquents après nettoyage}
\NormalTok{train\_tokens }\SpecialCharTok{\%\textgreater{}\%}
  \FunctionTok{count}\NormalTok{(word, }\AttributeTok{sort =} \ConstantTok{TRUE}\NormalTok{) }\SpecialCharTok{\%\textgreater{}\%}
  \FunctionTok{head}\NormalTok{(}\DecValTok{10}\NormalTok{) }\SpecialCharTok{\%\textgreater{}\%}
  \FunctionTok{ggplot}\NormalTok{(}\FunctionTok{aes}\NormalTok{(}\AttributeTok{x =} \FunctionTok{reorder}\NormalTok{(word, n), }\AttributeTok{y =}\NormalTok{ n)) }\SpecialCharTok{+}
  \FunctionTok{geom\_col}\NormalTok{(}\AttributeTok{fill =} \StringTok{"steelblue"}\NormalTok{) }\SpecialCharTok{+}
  \FunctionTok{coord\_flip}\NormalTok{() }\SpecialCharTok{+}
  \FunctionTok{theme\_minimal}\NormalTok{() }\SpecialCharTok{+}
  \FunctionTok{labs}\NormalTok{(}\AttributeTok{title =} \StringTok{"Top 10 des mots les plus fréquents (nettoyés)"}\NormalTok{,}
       \AttributeTok{x =} \StringTok{"Mots"}\NormalTok{, }\AttributeTok{y =} \StringTok{"Fréquence"}\NormalTok{)}
\end{Highlighting}
\end{Shaded}

\pandocbounded{\includegraphics[keepaspectratio]{Projet2_files/figure-latex/cleaning_logic-1.pdf}}

\subsection{2. Visualisation des termes
fréquents}\label{visualisation-des-termes-fruxe9quents}

\begin{Shaded}
\begin{Highlighting}[]
\CommentTok{\# Nuage de mots (Word Cloud)}
\FunctionTok{library}\NormalTok{(wordcloud)}

\CommentTok{\# Préparation des fréquences de mots}
\NormalTok{word\_counts }\OtherTok{\textless{}{-}}\NormalTok{ train\_tokens }\SpecialCharTok{\%\textgreater{}\%}
  \FunctionTok{count}\NormalTok{(word, }\AttributeTok{sort =} \ConstantTok{TRUE}\NormalTok{)}

\CommentTok{\# Génération du nuage de mots}
\FunctionTok{set.seed}\NormalTok{(}\DecValTok{1234}\NormalTok{) }\CommentTok{\# Pour la reproductibilité}
\FunctionTok{wordcloud}\NormalTok{(}\AttributeTok{words =}\NormalTok{ word\_counts}\SpecialCharTok{$}\NormalTok{word, }
          \AttributeTok{freq =}\NormalTok{ word\_counts}\SpecialCharTok{$}\NormalTok{n, }
          \AttributeTok{min.freq =} \DecValTok{50}\NormalTok{,           }\CommentTok{\# On n\textquotesingle{}affiche que les mots qui apparaissent au moins 50 fois}
          \AttributeTok{max.words =} \DecValTok{100}\NormalTok{,         }\CommentTok{\# Limite à 100 mots pour la lisibilité}
          \AttributeTok{random.order =} \ConstantTok{FALSE}\NormalTok{, }
          \AttributeTok{rot.per =} \FloatTok{0.35}\NormalTok{, }
          \AttributeTok{colors =} \FunctionTok{brewer.pal}\NormalTok{(}\DecValTok{8}\NormalTok{, }\StringTok{"Dark2"}\NormalTok{))}
\end{Highlighting}
\end{Shaded}

\pandocbounded{\includegraphics[keepaspectratio]{Projet2_files/figure-latex/unnamed-chunk-2-1.pdf}}

\begin{Shaded}
\begin{Highlighting}[]
\CommentTok{\# Top 10 des mots pour les tweets positifs vs négatifs}
\NormalTok{train\_tokens }\SpecialCharTok{\%\textgreater{}\%}
  \FunctionTok{filter}\NormalTok{(Sentiment }\SpecialCharTok{\%in\%} \FunctionTok{c}\NormalTok{(}\StringTok{"Positive"}\NormalTok{, }\StringTok{"Negative"}\NormalTok{)) }\SpecialCharTok{\%\textgreater{}\%}
  \FunctionTok{group\_by}\NormalTok{(Sentiment) }\SpecialCharTok{\%\textgreater{}\%}
  \FunctionTok{count}\NormalTok{(word, }\AttributeTok{sort =} \ConstantTok{TRUE}\NormalTok{) }\SpecialCharTok{\%\textgreater{}\%}
  \FunctionTok{slice\_max}\NormalTok{(n, }\AttributeTok{n =} \DecValTok{10}\NormalTok{) }\SpecialCharTok{\%\textgreater{}\%}
  \FunctionTok{ungroup}\NormalTok{() }\SpecialCharTok{\%\textgreater{}\%}
  \FunctionTok{ggplot}\NormalTok{(}\FunctionTok{aes}\NormalTok{(}\AttributeTok{x =} \FunctionTok{reorder\_within}\NormalTok{(word, n, Sentiment), }\AttributeTok{y =}\NormalTok{ n, }\AttributeTok{fill =}\NormalTok{ Sentiment)) }\SpecialCharTok{+}
  \FunctionTok{geom\_col}\NormalTok{(}\AttributeTok{show.legend =} \ConstantTok{FALSE}\NormalTok{) }\SpecialCharTok{+}
  \FunctionTok{facet\_wrap}\NormalTok{(}\SpecialCharTok{\textasciitilde{}}\NormalTok{Sentiment, }\AttributeTok{scales =} \StringTok{"free\_y"}\NormalTok{) }\SpecialCharTok{+}
  \FunctionTok{scale\_x\_reordered}\NormalTok{() }\SpecialCharTok{+}
  \FunctionTok{coord\_flip}\NormalTok{() }\SpecialCharTok{+}
  \FunctionTok{theme\_minimal}\NormalTok{() }\SpecialCharTok{+}
  \FunctionTok{labs}\NormalTok{(}\AttributeTok{title =} \StringTok{"Mots les plus fréquents par polarité"}\NormalTok{,}
       \AttributeTok{x =} \StringTok{"Mots"}\NormalTok{,}
       \AttributeTok{y =} \StringTok{"Fréquence"}\NormalTok{)}
\end{Highlighting}
\end{Shaded}

\pandocbounded{\includegraphics[keepaspectratio]{Projet2_files/figure-latex/unnamed-chunk-2-2.pdf}}

\subsection{4. Analyse comparative par
Entité}\label{analyse-comparative-par-entituxe9}

Le dataset contient des tweets portant sur différentes entités (marques
de tech, jeux vidéo). Cette section analyse comment le sentiment varie
d'une marque à l'autre. \#\#\# Top 10 des entités les plus actives
Voyons d'abord quelles sont les entités qui génèrent le plus de volume
de tweets dans notre base.

\begin{Shaded}
\begin{Highlighting}[]
\CommentTok{\# Calcul du volume par entité}
\NormalTok{entity\_counts }\OtherTok{\textless{}{-}}\NormalTok{ train\_data }\SpecialCharTok{\%\textgreater{}\%}
  \FunctionTok{count}\NormalTok{(Entity, }\AttributeTok{sort =} \ConstantTok{TRUE}\NormalTok{) }\SpecialCharTok{\%\textgreater{}\%}
  \FunctionTok{head}\NormalTok{(}\DecValTok{10}\NormalTok{)}

\FunctionTok{ggplot}\NormalTok{(entity\_counts, }\FunctionTok{aes}\NormalTok{(}\AttributeTok{x =} \FunctionTok{reorder}\NormalTok{(Entity, n), }\AttributeTok{y =}\NormalTok{ n, }\AttributeTok{fill =}\NormalTok{ Entity)) }\SpecialCharTok{+}
  \FunctionTok{geom\_col}\NormalTok{() }\SpecialCharTok{+}
  \FunctionTok{coord\_flip}\NormalTok{() }\SpecialCharTok{+}
  \FunctionTok{theme\_minimal}\NormalTok{() }\SpecialCharTok{+}
  \FunctionTok{guides}\NormalTok{(}\AttributeTok{fill =} \StringTok{"none"}\NormalTok{) }\SpecialCharTok{+}
  \FunctionTok{labs}\NormalTok{(}\AttributeTok{title =} \StringTok{"Top 10 des Entités les plus citées"}\NormalTok{,}
       \AttributeTok{x =} \StringTok{"Entité"}\NormalTok{,}
       \AttributeTok{y =} \StringTok{"Nombre de Tweets"}\NormalTok{)}
\end{Highlighting}
\end{Shaded}

\pandocbounded{\includegraphics[keepaspectratio]{Projet2_files/figure-latex/entity_volume-1.pdf}}

\begin{Shaded}
\begin{Highlighting}[]
\CommentTok{\# Sélection des 5 marques principales}
\NormalTok{top\_5\_entities }\OtherTok{\textless{}{-}} \FunctionTok{head}\NormalTok{(entity\_counts}\SpecialCharTok{$}\NormalTok{Entity, }\DecValTok{5}\NormalTok{)}

\NormalTok{train\_data }\SpecialCharTok{\%\textgreater{}\%}
  \FunctionTok{filter}\NormalTok{(Entity }\SpecialCharTok{\%in\%}\NormalTok{ top\_5\_entities) }\SpecialCharTok{\%\textgreater{}\%}
  \FunctionTok{ggplot}\NormalTok{(}\FunctionTok{aes}\NormalTok{(}\AttributeTok{x =}\NormalTok{ Entity, }\AttributeTok{fill =}\NormalTok{ Sentiment)) }\SpecialCharTok{+}
  \FunctionTok{geom\_bar}\NormalTok{(}\AttributeTok{position =} \StringTok{"fill"}\NormalTok{) }\SpecialCharTok{+} \CommentTok{\# Utilisation de "fill" pour avoir des pourcentages}
  \FunctionTok{theme\_minimal}\NormalTok{() }\SpecialCharTok{+}
  \FunctionTok{scale\_y\_continuous}\NormalTok{(}\AttributeTok{labels =}\NormalTok{ scales}\SpecialCharTok{::}\NormalTok{percent) }\SpecialCharTok{+}
  \FunctionTok{labs}\NormalTok{(}\AttributeTok{title =} \StringTok{"Proportion des sentiments pour les marques leaders"}\NormalTok{,}
       \AttributeTok{x =} \StringTok{"Marque / Jeu"}\NormalTok{,}
       \AttributeTok{y =} \StringTok{"Pourcentage"}\NormalTok{,}
       \AttributeTok{fill =} \StringTok{"Sentiment"}\NormalTok{) }\SpecialCharTok{+}
  \FunctionTok{theme}\NormalTok{(}\AttributeTok{axis.text.x =} \FunctionTok{element\_text}\NormalTok{(}\AttributeTok{angle =} \DecValTok{45}\NormalTok{, }\AttributeTok{hjust =} \DecValTok{1}\NormalTok{))}
\end{Highlighting}
\end{Shaded}

\pandocbounded{\includegraphics[keepaspectratio]{Projet2_files/figure-latex/entity_volume-2.pdf}}

\subsection{5. Comparaison Train vs
Validation}\label{comparaison-train-vs-validation}

Pour s'assurer de la robustesse de notre analyse, il est essentiel de
comparer la distribution des sentiments entre le jeu d'entraînement
(\texttt{train\_data}) et le jeu de validation (\texttt{val\_data}).
\#\#\# Alignement des distributions Nous vérifions si les proportions de
sentiments sont similaires dans les deux fichiers.

\begin{Shaded}
\begin{Highlighting}[]
\CommentTok{\# Préparation des données pour la comparaison}
\NormalTok{train\_dist }\OtherTok{\textless{}{-}}\NormalTok{ train\_data }\SpecialCharTok{\%\textgreater{}\%} 
  \FunctionTok{count}\NormalTok{(Sentiment) }\SpecialCharTok{\%\textgreater{}\%} 
  \FunctionTok{mutate}\NormalTok{(}\AttributeTok{Dataset =} \StringTok{"Training"}\NormalTok{, }\AttributeTok{prop =}\NormalTok{ n }\SpecialCharTok{/} \FunctionTok{sum}\NormalTok{(n))}

\NormalTok{val\_dist }\OtherTok{\textless{}{-}}\NormalTok{ val\_data }\SpecialCharTok{\%\textgreater{}\%} 
  \FunctionTok{count}\NormalTok{(Sentiment) }\SpecialCharTok{\%\textgreater{}\%} 
  \FunctionTok{mutate}\NormalTok{(}\AttributeTok{Dataset =} \StringTok{"Validation"}\NormalTok{, }\AttributeTok{prop =}\NormalTok{ n }\SpecialCharTok{/} \FunctionTok{sum}\NormalTok{(n))}

\NormalTok{comparison\_df }\OtherTok{\textless{}{-}} \FunctionTok{bind\_rows}\NormalTok{(train\_dist, val\_dist)}

\CommentTok{\# Visualisation comparative}
\FunctionTok{ggplot}\NormalTok{(comparison\_df, }\FunctionTok{aes}\NormalTok{(}\AttributeTok{x =}\NormalTok{ Sentiment, }\AttributeTok{y =}\NormalTok{ prop, }\AttributeTok{fill =}\NormalTok{ Dataset)) }\SpecialCharTok{+}
  \FunctionTok{geom\_col}\NormalTok{(}\AttributeTok{position =} \StringTok{"dodge"}\NormalTok{) }\SpecialCharTok{+}
  \FunctionTok{scale\_y\_continuous}\NormalTok{(}\AttributeTok{labels =}\NormalTok{ scales}\SpecialCharTok{::}\NormalTok{percent) }\SpecialCharTok{+}
  \FunctionTok{theme\_minimal}\NormalTok{() }\SpecialCharTok{+}
  \FunctionTok{labs}\NormalTok{(}\AttributeTok{title =} \StringTok{"Comparaison de la distribution des sentiments"}\NormalTok{,}
       \AttributeTok{subtitle =} \StringTok{"Vérification de la cohérence entre Training et Validation"}\NormalTok{,}
       \AttributeTok{y =} \StringTok{"Pourcentage du total"}\NormalTok{,}
       \AttributeTok{x =} \StringTok{"Catégorie de Sentiment"}\NormalTok{) }\SpecialCharTok{+}
  \FunctionTok{scale\_fill\_manual}\NormalTok{(}\AttributeTok{values =} \FunctionTok{c}\NormalTok{(}\StringTok{"Training"} \OtherTok{=} \StringTok{"\#5DADE2"}\NormalTok{, }\StringTok{"Validation"} \OtherTok{=} \StringTok{"\#EB984E"}\NormalTok{))}
\end{Highlighting}
\end{Shaded}

\pandocbounded{\includegraphics[keepaspectratio]{Projet2_files/figure-latex/comparison_logic-1.pdf}}

\subsection{6. Analyse de l'importance des mots (Log Odds
Ratio)}\label{analyse-de-limportance-des-mots-log-odds-ratio}

Pour terminer, nous allons identifier quels mots sont les plus
caractéristiques des sentiments ``Positive'' et ``Negative''. Cela
permet de comprendre ce qui influence réellement le score de sentiment.

\begin{Shaded}
\begin{Highlighting}[]
\CommentTok{\# Calcul du ratio d\textquotesingle{}apparition des mots entre Positif et Négatif}
\NormalTok{word\_ratios }\OtherTok{\textless{}{-}}\NormalTok{ train\_tokens }\SpecialCharTok{\%\textgreater{}\%}
  \FunctionTok{filter}\NormalTok{(Sentiment }\SpecialCharTok{\%in\%} \FunctionTok{c}\NormalTok{(}\StringTok{"Positive"}\NormalTok{, }\StringTok{"Negative"}\NormalTok{)) }\SpecialCharTok{\%\textgreater{}\%}
  \FunctionTok{count}\NormalTok{(word, Sentiment) }\SpecialCharTok{\%\textgreater{}\%}
  \FunctionTok{pivot\_wider}\NormalTok{(}\AttributeTok{names\_from =}\NormalTok{ Sentiment, }\AttributeTok{values\_from =}\NormalTok{ n, }\AttributeTok{values\_fill =} \DecValTok{0}\NormalTok{) }\SpecialCharTok{\%\textgreater{}\%}
  \FunctionTok{filter}\NormalTok{(Positive }\SpecialCharTok{\textgreater{}} \DecValTok{10} \SpecialCharTok{\&}\NormalTok{ Negative }\SpecialCharTok{\textgreater{}} \DecValTok{10}\NormalTok{) }\SpecialCharTok{\%\textgreater{}\%} \CommentTok{\# On garde les mots fréquents}
  \FunctionTok{mutate}\NormalTok{(}\AttributeTok{log\_ratio =} \FunctionTok{log}\NormalTok{(Positive }\SpecialCharTok{/}\NormalTok{ Negative)) }\SpecialCharTok{\%\textgreater{}\%}
  \FunctionTok{arrange}\NormalTok{(}\FunctionTok{desc}\NormalTok{(log\_ratio))}

\CommentTok{\# Affichage des mots les plus distinctifs}
\FunctionTok{bind\_rows}\NormalTok{(}
\NormalTok{  word\_ratios }\SpecialCharTok{\%\textgreater{}\%} \FunctionTok{slice\_max}\NormalTok{(log\_ratio, }\AttributeTok{n =} \DecValTok{10}\NormalTok{), }\CommentTok{\# Top Positifs}
\NormalTok{  word\_ratios }\SpecialCharTok{\%\textgreater{}\%} \FunctionTok{slice\_min}\NormalTok{(log\_ratio, }\AttributeTok{n =} \DecValTok{10}\NormalTok{)  }\CommentTok{\# Top Négatifs}
\NormalTok{) }\SpecialCharTok{\%\textgreater{}\%}
  \FunctionTok{mutate}\NormalTok{(}\AttributeTok{word =} \FunctionTok{reorder}\NormalTok{(word, log\_ratio)) }\SpecialCharTok{\%\textgreater{}\%}
  \FunctionTok{ggplot}\NormalTok{(}\FunctionTok{aes}\NormalTok{(}\AttributeTok{x =}\NormalTok{ word, }\AttributeTok{y =}\NormalTok{ log\_ratio, }\AttributeTok{fill =}\NormalTok{ log\_ratio }\SpecialCharTok{\textgreater{}} \DecValTok{0}\NormalTok{)) }\SpecialCharTok{+}
  \FunctionTok{geom\_col}\NormalTok{() }\SpecialCharTok{+}
  \FunctionTok{coord\_flip}\NormalTok{() }\SpecialCharTok{+}
  \FunctionTok{scale\_fill\_manual}\NormalTok{(}\AttributeTok{values =} \FunctionTok{c}\NormalTok{(}\StringTok{"tomato"}\NormalTok{, }\StringTok{"skyblue"}\NormalTok{), }\AttributeTok{labels =} \FunctionTok{c}\NormalTok{(}\StringTok{"Négatif"}\NormalTok{, }\StringTok{"Positif"}\NormalTok{)) }\SpecialCharTok{+}
  \FunctionTok{theme\_minimal}\NormalTok{() }\SpecialCharTok{+}
  \FunctionTok{labs}\NormalTok{(}\AttributeTok{title =} \StringTok{"Mots les plus spécifiques par sentiment"}\NormalTok{,}
       \AttributeTok{subtitle =} \StringTok{"Log odds ratio (Positif vs Négatif)"}\NormalTok{,}
       \AttributeTok{y =} \StringTok{"Log Ratio (Positif \textless{}{-}{-}{-}\textgreater{} Négatif)"}\NormalTok{,}
       \AttributeTok{x =} \StringTok{"Mots"}\NormalTok{,}
       \AttributeTok{fill =} \StringTok{"Tendance"}\NormalTok{)}
\end{Highlighting}
\end{Shaded}

\pandocbounded{\includegraphics[keepaspectratio]{Projet2_files/figure-latex/log_odds-1.pdf}}

\end{document}
